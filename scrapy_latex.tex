\documentclass{beamer}
%\usepackage{lmodern}
\usepackage[ngerman]{babel}
\usepackage[utf8]{inputenc}
\usepackage{listings}

% lstlisting-Einstellungen
\lstset{
	language=Python,
	morekeywords = {yield},
	basicstyle=\ttfamily,
	keywordstyle=\color{blue}\ttfamily,
	stringstyle=\color{red}\ttfamily,
	commentstyle=\color{green}\ttfamily,
	morecomment=[l][\color{magenta}]{\#},
	% wegen Leerzeichen-Anzeige	
	showstringspaces=false,
	% wegen Character-Spacing
	columns=fullflexible,	
}

% auto-nummerierung bei framebreaks
\setbeamertemplate{frametitle continuation}{(\insertcontinuationcount)}

% etwas nach links
\setlength{\leftmargini}{1em}

\title{Scrapy}
\subtitle{A Fast and Powerful Scraping and Web Crawling Framework}
\author{LdPx, lammbraten, foobar999}
\subject{Computer Science}

\begin{document}

\frame{\titlepage}
\begin{frame}
	\frametitle{Übersicht}
	\tableofcontents
\end{frame}
\section{Allgemeines}
\begin{frame}
	\frametitle{\insertsection}
	\begin{itemize}
		\item  Definition Scraping...
		\item  Was ist Web-Scraping?
		\item  Framework für Python
		\item  Crawlen von Websites
		\item  Extraktion strukturierter Daten
	\end{itemize} 
\end{frame}

\section{Installation}
% fragile nötig für listings
\begin{frame}[fragile]
	\frametitle{\insertsection}
	\begin{itemize}
		\item mit pip (erfordert vorhandene Python-Installation):
			\begin{lstlisting}
pip3 install scrapy
			\end{lstlisting}
		\item mit anaconda (in Windows einfacher, da numpy, etc... bereits enthalten):
			\begin{lstlisting}
conda install -c conda-forge scrapy
			\end{lstlisting}
	\end{itemize}
\end{frame}

\section{Projekt anlegen}
\begin{frame}[fragile]
	\frametitle{\insertsection}
	\begin{itemize}
		\item Verzeichnisstruktur erzeugen:
			\begin{lstlisting}
scrapy startproject mynewproject
			\end{lstlisting}
		\item erzeugt ein gleichnamiges Verzeichnis \verb|./mynewproject| 
		\item künftige Spider werden in \verb|./mynewproject/spiders| angelegt
		\item Konfiguration erfolgt über \verb|./mynewproject/settings.py|
	\end{itemize}
\end{frame}

\section{Spider}
\begin{frame}[fragile]
	\frametitle{\insertsection}
	\begin{itemize}
		\item spezielle Klassen in Scrapy-Projekten
		\item von Klasse \verb|scrapy.Spider| abgeleitet
		\item führen Crawling durch, spezifisch für Websites programmierbar
		\item Ablauf: \begin{enumerate}		
			\item Spider schickt Requests an initiale URLs
			\item scrapy ruft je Response Callback-Methode auf, mit Inhalt als Parameter
			\item Callback-Methode startet ggf. Requests an weitere URLs
			\item Callback-Methode extrahiert Daten des Response und gibt sie zurück
			\item scrapy sammelt alle zurückgegebenen Daten ein, und speichert sie z.B. in einer Datei
		\end{enumerate}
	\end{itemize}
\end{frame}
\begin{frame}[fragile,allowframebreaks]
	\frametitle{einfacher Beispielspider}
			\begin{lstlisting}
import scrapy

class SimpleSpider(scrapy.Spider):
    name = 'simplespider'
    start_urls = ['http://supersimpleloremipsum.com/']

    def parse(self, response):
        self.logger.info('parsing {}'.format(response))
        yield {'status': response.status}
			\end{lstlisting}
			
	\framebreak
	
	\begin{itemize}
		\item \lstinline|name| identifiziert den Spider eindeutig im Projekt
		\item \lstinline|start_urls| beinhaltet die URLs für die initialen Requests
		\item \lstinline|parse()| wird als Callback aufgerufen, der Parameter \lstinline|response| enthält das Ergebnis des Response
		\item die Methode erzeugt eine Logging-Ausgabe von \lstinline|response| 
		\item die Methode übergibt scrapy das Parsingergebnis den HTTP-Statuscode als \lstinline|dict|
	\end{itemize}
	
	\framebreak
	
	\begin{itemize}
		\item das Crawling mit Spider \lstinline|simplespider| wird über einen eigenen Befehl in der Kommandozeile gestartet:
			\begin{lstlisting}
scrapy crawl simplespider -o res.json
			\end{lstlisting}
		\item scrapy speichert die per \lstinline|yield| zurückgegebenen Daten in der Datei \lstinline|res.json| (das Dateiformat ergibt sich aus Suffix) \begin{itemize}
			\item scrapy unterstützt neben json weitere Formate: xml, csv, ... 
			\item beachte: scrapy leert bei mehrfachem Start des Crawlingbefehls die Datei nicht, sondern hängt die Daten hinten dran
		\end{itemize}
		\item Befehl auch über Pythonskript aufrufbar: \begin{lstlisting}
def main():
  cmd = 'scrapy crawl iter -o res.json'
  scrapy.cmdline.execute(cmd.split())
main()		
		\end{lstlisting}
		\item API zum Crawlen: Fehlanzeige ?
	\end{itemize}
\end{frame}

\section{Extraktion von Daten}
\begin{frame}[fragile,allowframebreaks]
	\frametitle{\insertsection{}}
	\begin{lstlisting}[breaklines=true]
def parse(self, resp):
  res1 = resp.css('div small')
  #[<Selector data='<small...">'>, <Selector data='<small...">'>, ...]
  res2 = resp.css('div small::text')
  #[<Selector data='Albert Einstein'>, <Selector data='J.K. Rowling'>, ...]
  res3 = resp.css('div small::text').extract()
  #['Albert Einstein', 'J.K. Rowling', ...]
  res4 = resp.css('div small::text').extract_first()
  #Albert Einstein
  ...
	\end{lstlisting}
	
	\framebreak
	
	\begin{itemize}
		\item \lstinline|resp.css('div small')| selektiert mehrere Elemente im DOM mit dem CSS-Selektor \lstinline|'div small'| (d.h. es findet alle \lstinline|<small>|-Elemente innerhalb von \lstinline|<div>|-Elementen)
		\item \lstinline|resp.css('div small::text')| selektiert den Text innerhalb jedes gefundenen \lstinline|<small>...</small>|-Tags
		\item \lstinline|resp.css('div small::text').extract()| liefert die selektierten Texte als Liste von Python-Strings
		\item \lstinline|resp.css('div small::text').extract_first()| liefert den 1. Treffer davon
	\end{itemize}
	
	\framebreak
	
	\begin{itemize}
		\item das Ergebnis einer Selektion kann selbst wieder zum Selektieren genutzt werden
		\item folgendes Beispiel selektiert alle Zitate einer Website, und gibt je Zitat als Datensatz Autor und Wortlaut zurück:
	\begin{lstlisting}
def parse(self, response):
  quotes = response.css('div.quote')
  for q in quotes:
    yield {
      'author': q.css('.author::text').extract_first(),
      'text': q.css('.text::text').extract_first()
    }
	\end{lstlisting}
	\end{itemize}
\end{frame}

\section{Links folgen}
\begin{frame}[fragile,allowframebreaks]
	\frametitle{\insertsection{}}
	\begin{lstlisting}
def parse(self, response):
  for q in response.css('div.quote'):
    yield {
      'text': q.css('.text::text').extract_first(),
      'author': q.css('.author::text').extract_first()
    }
  a_selector = 'li.next a::attr(href)'
  hrefs = response.css(a_selector).extract()  
  for href in hrefs:    
    yield response.follow(href, callback=self.parse)
	\end{lstlisting}
	\begin{itemize}
		\item die Methode liefert zunächst von den Zitaten Autor und Wortlaut
		\item danach sucht sie alle passenden \lstinline|<a href=...>|-Elemente und selektiert je Treffer das \lstinline|href|-Attribut mit \lstinline|'a::attr(href)'| 
		
		\framebreak
		
		\item zuletzt extrahiert die Methode die enthaltenen URLs und untersucht sie rekursiv mit \lstinline|response.follow(href, callback=self.parse)| 
		\item scrapy merkt sich außerdem intern schon besuchte URLs und vermeidet so mögliche Endlosrekursionen durch wechselseitige URL-Verweise
	\end{itemize}
\end{frame}

\section{Quellen}
\begin{frame}
	\frametitle{\insertsection{}}
	% lustiges welt-icon
	\emergencystretch 1.5em
	\setbeamertemplate{bibliography item}{\insertbiblabel}
	{\small
	\begin{thebibliography}{8}
		\bibitem{scrapy tu}\url{https://doc.scrapy.org/en/latest/intro/tutorial.html}
		\bibitem{scrapy ins}\url{https://doc.scrapy.org/en/latest/intro/install.html}
		\bibitem{scrapy sp}\url{https://doc.scrapy.org/en/latest/topics/spiders.html}
	\end{thebibliography}
	}
\end{frame}



\section{�bung}

\subsection{-> Emails}
\begin{frame}[fragile,allowframebreaks]
\end{frame}

\subsection{Blog-Posts}
\begin{frame}[fragile,allowframebreaks]
\end{frame}

\subsection{Link-counter}
\begin{frame}[fragile,allowframebreaks]
\end{frame}


\end{document}